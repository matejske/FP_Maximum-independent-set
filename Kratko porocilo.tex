\documentclass[a4paper, 12 pt]{article}
\usepackage[utf8]{inputenc}
\usepackage[T1]{fontenc}
\usepackage[slovene]{babel}
\usepackage{lmodern}
\usepackage{amsmath}
\usepackage{amsfonts}
\usepackage{amssymb}
\usepackage{units}
\usepackage{eurosym}
\usepackage{pdfpages}
\usepackage{comment}
\usepackage{enumerate}
\usepackage{mathtools}
\usepackage{amsthm}

\theoremstyle{plain}
\newtheorem{izrek}{Izrek}
\theoremstyle{definition}
\newtheorem{definicija}{Definicija}

\begin{document}
\begin{titlepage}
		\begin{center}
		
		\large
		Univerza v Ljubljani\\
		\normalsize
		Fakulteta za matematiko in fiziko\\
		
		\small
		Finančna matematika - 1. stopnja\\
		
		\vspace{5 cm} 
		
		\large
		Matej Škerlep \\
		
		\vspace{0.5cm}
		\Large
		\textbf{Problem največje množice neodvisnih vozlišč (kratko poročilo)}
		
		\vspace{0.5 cm}
		\normalsize
		Finančni praktikum
		
		\vspace{1.5cm}
		\normalsize
		Mentorja: prof. dr. Riste Škrekovski in asist. dr. Janoš Vidali
		
		\vfill
		
		\large Ljubljana, 2019
		
		\end{center}
\end{titlepage}


\section{Navodila za delo}
\begin{itemize}
\item Definirajte problem največje množice nesosednjih vozlišč kot CLP in ga rešite za nekaj primerov. 
\item Eksperimentalno rimerjajte rezultate CLP in njegove relaksacije na LP in ugotovite, za koliko se lahko razlikujejo med sabo po velikosti.
\item Napišite algoritem za lokalno iskanje po grafu in njegov rezultate primerjajte s prejšnjimi. 
\item Ugotovite za kako velike grafe je posamezen izmed primerov rešljiv.
\end{itemize}

\section{Definicija pojma}

\begin{definicija}Naj bo $G = (V, E)$ graf in $I \subseteq V$. Množica vozlišč $I$ je \textbf{\textit{neodvisna}}, če ne vsebuje sosednjih vozlišč. 
\newline Formalno, če za $\forall v, u \in V$, $uv \in E$ velja: $v \in I \Leftrightarrow  u \notin I$ \end{definicija}



\section{Plan dela}
\subsection{Celoštevilski linearen program}
Celoštevilski linearen program za dani problem se glasi:


\begin{equation*}
\begin{array}{ll@{}ll}
\text{max}  & \displaystyle\sum\limits_{v\in V} x_v &&\\
\text{p.p.}	 & x_u + x_v \leq 1,		 			&&\text{za vsak par} uv \in E \\
                 & x_v \in \{0,1\}, 	              			&&\forall v \in V
\end{array}
\end{equation*}



\subsection{Ideja za lokalno iskanje}
začnemo z množico nesosednjih vozlišč (naj bo to recimo $I$) in nato naključno zamenjamo eno iz vozlišč iz množice $I$ z vozliščem, ki ga v nožici ni. Pri tem upamo, da bo po zadosti menjavah eno izmed vozlišč postalo prosto. Torej ga lahko dodamo v množico $I$ in s tem njeno moč povečamo za 1.





\end{document}